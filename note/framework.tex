\section{Framework}

    This section provides instructions for using the analysis framework. The framework includes the following macros:
    \begin{itemize}
        \item \url{../convert_macro/convertBmn_run8.C} and \url{../convert_macro/run8_convert.sh}: These files are conversion macros used to convert files from the .dst format to a ROOT tree (.tree.root).
        \item \url{../QA_macro/run8_qa_new.C}:This is the main macro for the QA (Quality Assurance) stage. It defines the variables for analysis and sets up the corresponding histograms. \url{../QA_macro/run8.sh}: This script is used to run the  \url{../run8_qa_new.C} macro on the NICA cluster.
        \item \url{../QA_macro/GoldRuns_script.C}: This macro is used for drawing figures and run-by-run QA analysis.
        \item \url{../QA_macro/Get_VtxXYZ_corrRunId.C}: This macro is used for run-by-run correction of the x, y and z positions of the reconstructed vertex.
        \item \url{../QA_macro/Get_GraphiCuts_cuts.C}: This macro defines a graphical cut to reduce the impact of pileup events based on the number of FSD and GEM digits and multiplicity.
        \item \url{../QA_macro/Get_BC1_FD_cuts.C}: This macro defines a graphical cut to reduce the impact of pileup events based on signal data from the BC1 and FD detectors.
        \item \url{../QA_macro/refMult_corr.C}: This macro is used for run-by-run correction of the multiplicity of charged particles.
    \end{itemize}
    Detailed instructions will be provided for each macro below.

\subsection{convertBmn\_run8.C macro}\label{convert}

    \url{convert_macro/convertBmn_run8.C} - This macro converts files from the .dst and .digi formats into a single simple format: a ROOT tree. 
    The input files are .dst and .digi files from EOS, as well as .hitInfo.root files. The latter contain the results of the pileup event analysis conducted by Oleg Golosov.
    To start on a cluster, use the \url{convert_macro/run8_convert.sh} script. 
    Example of running a script: sbatch \url{run8_convert.sh} \url{path_to_lists/name_list.list} \url{out_dir}, 
    where one \url{name_list.list} contains the names of 999 files. From now on we will work only with the our .tree.root format using the ROOT's RDataFrame.
    Example of running a macro for 1 file:
    
    root -l -q -b convertBmn\_run8.C'("iFile\_dst","iFile\_digi","iFile\_hitInfo","oFile")', 
    where iFile\_dst, iFile\_digi, iFile\_hitInfo - full path to file format dst, digi and hitInfo respectively.  

\subsection{run8\_qa\_new.C macro}\label{qaMacro}

    \url{../QA_macro/run8_qa_new.C} - This macro is used to read the .tree format and create histograms. Additionally, you can add your own variables and their corresponding histograms. \url{../QA_macro/run8.sh}: This script is used to run the  \url{../run8_qa_new.C} macro on the NICA cluster.
    The input parameters:
    \begin{itemize}
        \item str\_in\_list - list of .tree.root files
        \item str\_in\_list\_plp - list of files containing pileup corrections.
        \item out\_file\_name - output file name
        \item in\_fit\_file - this file containing corrections for the vertex and multiplicity. This file is specified during the second run of the \url{../QA_macro/run8_qa_new.C}. The correction file is obtained after running the \url{refMult_corr.C} and \url{Get_VtxXYZ_corrRunId.C} macros.
    \end{itemize}
    The output is a file containing a set of histograms. Example of running a macro for 1 list with files:

    root -l -q -b run8\_qa\_new.C'("iList","","oFile","iCorrFile")', where iList - input list with tree.root files, oFile - output file, iCorrFile - file with corrections of vertex and multiplicity , indicate at the second launch.


\subsection{GoldRuns\_script.C macro}\label{BadRun}
    
    \url{../QA_macro/GoldRuns_script.C}: This macro is used for drawing figures and run-by-run QA analysis.
    The input parameters:
    \begin{itemize}
        \item \_inFileName - input file name
        \item \_outFileName - output file name
    \end{itemize}
    After the macro runs, a set of specified plots and a list of bad runs will be generated.
    Example of running a macro:

    root -l -q -b GoldRuns\_script.C

\subsection{Get\_VtxXYZ\_corrRunId.C macro}\label{corrVtx}

    \url{../QA_macro/Get_VtxXYZ_corrRunId.C}: This macro is used for run-by-run correction of the x, y and z positions of the reconstructed vertex.
    The input parameters:
    \begin{itemize}
        \item \_file\_inFile - input file name
        \item \_file\_outFile - output file name
    \end{itemize}
    The result of running the macro will be a file containing 2D histograms. 
    The x-axis represents the run number, and the y-axis represents the value by which the reconstructed vertex needs to be corrected.
    Example of running a macro:
    
    root -l -q -b Get\_VtxXYZ\_corrRunId.C


\subsection{Get\_GraphiCuts\_cuts.C macro}\label{GraphCut1}

    \url{../QA_macro/Get_GraphiCuts_cuts.C}: This macro defines a graphical cut to reduce the impact of pileup events based on the number of FSD and GEM digits and multiplicity.
    \begin{itemize}
        \item \_file\_inFile - input file name
        \item \_file\_outFile - output file name
    \end{itemize}
    After the macro runs, we obtain the parameters for a graphical cut that reduces the contribution of pileup events. The result of this macro needs to be manually recorded in \url{../QA_macro/run8_qa_new.C} in functions $stsNdigitsMultCut$
    Example of running a macro:
    
    root -l -q -b Get\_GraphiCuts\_cuts.C



\subsection{Get\_BC1\_FD\_cuts.C macro}\label{GraphCut2}

    \url{../QA_macro/Get_BC1_FD_cuts.C} - macro defines a graphical cut run-by-run to reduce the impact of pileup events based on signal data from the BC1 and FD detectors.
    \begin{itemize}
        \item \_file\_inFile - input file name
        \item \_file\_outFile - output file name
    \end{itemize}
    After the macro runs, graphs for run-by-run corrections of signals from the BC1 and FD detectors will be written to the output file. 
    These corrections are used when macro \url{../QA_macro/run8_qa_new.C} is run again.
    Example of running a macro:
    
    root -l -q -b Get\_BC1\_FD\_cuts.C



\subsection{refMult\_corr.C macro}\label{CorrRefMult}

   \url{../QA_macro/refMult_corr.C}: This macro is used for run-by-run correction of the multiplicity of charged particles.
    \begin{itemize}
        \item \_file\_inFile - input file name
        \item \_file\_outFile - output file name
        \item Specify the region of stable runs to which the multiplicity from other runs will be fitted.
    \end{itemize}
    After the macro runs, graphs for run-by-run corrections for multiplicity will be written to the output file.
    These corrections are used when macro \url{../QA_macro/run8_qa_new.C} is run again.

    Example of running a macro:
    
    root -l -q -b refMult\_corr.C

    The procedure for multiplicity corrections consists of the following steps:
    \begin{itemize}
        \item We used events from Physical runs and CCT2 trigger.
        \item remove "bad" runs
        \item remove events with pileup
        \item apply event selection criteria: at least 2 tracks in vertex reconstruction
        \item Extract the high-end point of refMult distribution in each RunId via fitting the refMult tail by the function:
            \begin{equation}\label{ErrFunct}
                f(refMult)=A*Erf(-\sigma(refMult-H))+A
            \end{equation}
        \item refMult can then be corrected by:
            \begin{equation}\label{ErrFunct}
                refMult_{corr} = refMult * H_{ref} / H(RunId),
            \end{equation}
        where $H_{ref}$ - high-end point of refMult distribution from stable runs. 


    \end{itemize}
